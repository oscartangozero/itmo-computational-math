\documentclass[report.tex]{subfiles}

\begin{document}
    Cистемой из $m$ линейных алгебраических уравнений назвают совокупность уравнений,
    использующих один и тот же набор из $n$ неизвестных:
    \[\begin{aligned}
          a_{11}x_{1}+a_{12}x_{2}+ \cdots +a_{1n}x_{n} & =b_{1} \\
          a_{21}x_{1}+a_{22}x_{2}+ \cdots +a_{2n}x_{n} & =b_{2} \\
          &\ \ \vdots \\
          a_{m1}x_{1}+a_{m2}x_{2}+ \cdots +a_{mn}x_{n} & =b_{m},
    \end{aligned}\]
    где $x_{1},x_{2},\ldots,x_{n}$ -- неизвестные, $a_{11},a_{12},\ldots,a_{mn}$ -- коэффициенты системы,
    $b_{1},b_{2},\ldots,b_{m}$ -- свободные члены.

    Такая система эквивалентна следующему матричному уравнению:
    \[Ax=b\]
    \[A={\begin{bmatrix}
             a_{11} & a_{12} & \cdots & a_{1n} \\
             a_{21} & a_{22} & \cdots & a_{2n} \\
             \vdots & \vdots & \ddots & \vdots \\
             a_{m1} & a_{m2} & \cdots & a_{mn}
    \end{bmatrix}}, \quad
    x = {\begin{bmatrix}
             x_{1} \\ x_{2} \\ \vdots \\x_{n}
    \end{bmatrix}}, \quad
    b ={\begin{bmatrix}
            b_{1} \\ b_{2} \\ \vdots \\ b_{m}
    \end{bmatrix}}\]

    Интересным представляется случай, когда $n=m, \ \det{A} \ne 0$ и система имеет единственное решение.
    Одним из методов приближенного решения таких систем является \textbf{метод простых итераций} (он же метод Якоби).

    Метод предполагает разложение матрицы коэффициентов:
    \[A=D+N, \quad \text{где} \quad
    D={\begin{bmatrix}
           a_{11} & 0      & \cdots & 0      \\
           0      & a_{22} & \cdots & 0      \\
           \vdots & \vdots & \ddots & \vdots \\
           0      & 0      & \cdots & a_{nn}
    \end{bmatrix}}
        {\text{ и }}
    N={\begin{bmatrix}
           0      & a_{12} & \cdots & a_{1n} \\
           a_{21} & 0      & \cdots & a_{2n} \\
           \vdots & \vdots & \ddots & \vdots \\
           a_{n1} & a_{n2} & \cdots & 0
    \end{bmatrix}}\]

    При подстановке получим реккурентную формулу последовательности приближений неизвестных:
    \[x^{(k+1)} = D^{-1}(b - N x^{(k)}) = x^{(k)} + D^{-1}(b - A x^{(k)})\]

    Достаточным условием сходимости такой последовательности является:
    \[\rho(D^{-1}N) < 1 \quad \Leftrightarrow \quad \lim_{k \rightarrow \infty} (D^{-1}N)^k = 0\]

    Условие более строгое, но которые проще проверить -- свойство строгого диагонального преобладания матрицы $A$:
    \[|a_{ii}| \geq \sum _{j\neq i}|a_{ij}| \quad \forall i\]
\end{document}