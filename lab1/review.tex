\documentclass[lab1-report.tex]{subfiles}

\begin{document}
    Метод Якоби относится к группе итеративных методов решения определенных систем линейных уравнений:
    По сравнению с традиционными методами решения (метод Гаусса и его вариации) итерационные методы в целом:
    \begin{itemize}
        \item позволяют получить решение с произвольной точностью, избегая накапливание ошибки между итерациями;
        \item могут быть использованы для уточнения уже имеющегося приближения;
        \item требуют лишь $O(N)$ памяти для хранения промежуточного состояния;
        \item предоставляют возможности параллелизации вычислений в рамках одной итерации;
        \item крайне зависимы от входных данных: имеют зависимую от скорости схождения временную асимптотику
        -- $O(steps \cdot N^2)$;
        \item также весьма требовательны к входным данным -- значительное количество матриц не удовлетворяют
        условию сходимости.
    \end{itemize}
    Метод простых итераций является простейшим в своей категории.
    Существуют его модификации -- например метод Гаусса-Зейделя, который:
    \begin{itemize}
        \item обеспечивает б\'{о}льшую скорость сходимости за счет использования информации
        об уже подсчитанных приближениях в рамках одной итерации;
        \item по той же причине усложняет параллелизацию;
        \item сокращает потребность в памяти на $O(N)$ (потенциально в 2 раза).
    \end{itemize}
\end{document}